\documentclass[english]{article}
\usepackage[T1]{fontenc}
\usepackage[utf8]{inputenc}
\usepackage{listings}
\usepackage{graphicx}
\usepackage{epstopdf}
\usepackage{etex}
\reserveinserts{100}
\usepackage{morefloats}
\usepackage{dcolumn}
\usepackage{tabularx}
\usepackage{multirow}
\usepackage{array}
\usepackage{chngpage}
\usepackage{booktabs}
\usepackage[spanish]{babel}
\usepackage{verbatim}
\usepackage{moreverb}
\usepackage{color}

\decimalpoint

\let\verbatiminput=\verbatimtabinput
\def\verbatimtabsize{4\relax}

\renewcommand{\tablename}{Tabla}

\newcolumntype{K}{>{\centering\arraybackslash$}X<{$}}


\begin{document}

\makeatletter
\title{Multi-Objective Compact Differential Evolution}
\author{Moisés Osorio}
\date{\today}
\makeatother
\maketitle

\section*{Resultados}
Varias funciones de prueba se han utilizado para comparar los resultados obtenidos por la implementación actual de mocDE. Las siguientes figuras muestran el mejor y peor desempeño alcanzado por mocDE en comparación con el frente de Pareto real. Así mismo, las tablas muestran la media y la desviación estándar de los resultados de algunas métricas calculadas sobre 30 ejecuciones de mocDE, MOEA/D, PAES y NSGA-II. En la tabla \ref{tab:results-summary} se puede visualizar el resumen de todos los resultados obtenidos.

El número máximo de evaluaciones hechas por cada ejecución es 300,000 para la familia de funciones UF del CEC'2009 y 20,000 para las demás funciones.

Los parámetros utilizados para las funciones de prueba son los detallados en la tabla \ref{tab:params}. Donde $n_{var}$ es el número de variables del problema.

\begin{table}
	\centering
        \begin{tabularx}{1\textwidth}{| c || c | K |}
        \hline
            Algoritmo & Parámetro & Valor \\ \hline \hline
            \multirow{4}{*}{\textbf{mocDE}}
                & Tamaño de población $p$ & 100 \\
                & Variación diferencial $F$ & 1.0 \\
                & Probabilidad de cruza $C$ & 0.1 \\
                \hline
            \multirow{8}{*}{\textbf{MOEA/D}}
                & Tamaño de población $p$ & 100 \\
                & Tamaño de nicho & 100 (2D), 150 (3D) \\
                & Límite de actualización & 10 (2D), 15 (3D) \\
                & Variación diferencial $F$ & 1.0 \\
                & Probabilidad de cruza $C$ & 0.5 \\
                & Tasa de mutación $m$ & 1.0 / n_{var} \\
                & Mating selection probability & 0.9 \\
                \hline
            \multirow{6}{*}{\textbf{PAES}}
                & Tamaño de archivo $p$ & 100 \\
                & Tasa de mutación $m$ & 1.0 / n_{var} \\
                & Bisección & 5 \\
                & Índice de distribución & 20 \\
                \hline
            \multirow{4}{*}{\textbf{NSGA-II}}
                & Tamaño de población $p$ & 100 \\
                & Tasa de mutación $m$ & 0.01 \\
                & Probabilidad de cruza $C$ & 0.9 \\
                \hline
        \end{tabularx}
    \caption{\label{tab:params} Parámetros de los algoritmos.}
\end{table}

%RESULTS%

\end{document}
